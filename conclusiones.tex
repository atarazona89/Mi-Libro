\chapter{Conclusiones y Recomendaciones}

    Habiendo terminado el proyecto, se logró la implementación del sistema ``HxPlus Ocupacional" para Globisoft S.A. apoyando de esta manera su misión de brindar apoyo tecnológico al área médica en Venezuela. El mercado actual de herramientas en este ámbito está en su fase inicial y, aunque hay resistencia al cambio por parte de los médicos nacionales, la familiaridad y una interfaz planificada para la usabilidad, ayudará a que las nuevas generaciones hagan uso de este sistema.
    
    SCRUM permitió una organización rápida y la debida evaluación oportuna de los objetivos logrados y la planificación de la siguiente serie de objetivos, adaptándose así a los inconvenientes evidenciados.
    
    El proyecto, si bien orientado a medicina ocupacional, se recomienda en un futuro, incluir también el área farmacéutica y su integración con ``HxPlus", el cual se encuentra en funcionamiento. Con esto se crearía un sistema distribuido integral de gestión de consultas, remisión de informes médicos y generación de constancias, informes y récipes médicos y que a su vez le permita a las empresas farmacéuticas publicidad y registros en tiempo real, a los pacientes, tener siempre a su disposición los planes de tratamiento, récipes médicos, los diagnósticos realizados y su historial médico en caso de alguna eventualidad como pérdida del mismo u olvidos ocasionales.
    
    También, y en el marco de las tecnologías usadas, se sugiere una evaluación del patrón de almacenamiento de la base de datos ya que el uso de un patrón orientado a eventos podría ser provechoso al momento de recuperación de fallos dentro de la base de datos.


