\chapter{Marco Tecnológico}

    \section{Herramientas para el desarrollo de la aplicación}
    
    En esta sección se describen las herramientas usadas en el desarrollo de la aplicación y las características que hicieron que fueran seleccionadas para tal fin.
    
        \subsection{Eclipse}
    
        \subsection{Java}
        
        Lenguaje de programación utilizado para el desarrollo del \textit{backend}. Java es un lenguaje de programación imperativo orientado a objetos que facilita el desarrollo intependiente de los servicios y el fácil acceso al modelo de datos permitiendo así la baja cohesión entre los componentes del sistema.
        
        \subsection{JSON}
        
        Por \textit{JavaScript Object Notation}, es un formato de intercambio de datos, ligero\cite{JSON-yahoo} que permite una sencilla comunicación entre la vista y el controlador.
        
        "JSON es un formato de texto que es completamente independiente del lenguaje pero utiliza convenciones que son ampliamente conocidos por los programadores de la familia de lenguajes C, incluyendo C, C++, C\#, Java, JavaScript, Perl, Python, y muchos otros. Estas propiedades hacen que JSON sea un lenguaje ideal para el intercambio de datos\cite{JSON-jsonOrg}."
        
        Fue elegido por su compatibilidad con Java y porque es independiente de la tecnología usada en la Vista, lo cual permite a su vez realizar cambios en la Vista sin afectar las funcionalidades del controlador.
        
        \subsection{Angular JS}
        
        Es un \textit{framework} o conjunto de librerias orientada a facilitar el desarrollo web de aplicaciones dinámicas. "AngularJS le permite extender el vocabulario HTML para su aplicación"\cite{ANGULARJS-angularjs}.
        
        \subsection{JavaScript}
        
        JavaScript (abreviado comúnmente "JS") es un lenguaje de programación interpretado, dialecto del estándar ECMAScript. Se define como orientado a objetos, basado en prototipos, imperativo, débilmente tipado y dinámico\cite{JAVASCRIPT-wiki}.
        
        Es un lenguaje de programación orientado a crear contenidos dinámicos para páginas web del lado del cliente. Tanto JSON como AngularJS, previamente mencionados, usan JavaScript en su codificación.
        
        \subsection{MySQL}
        
        Manejador de bases de datos, \textit{open source}, ofrece alto rendimiento, eficiencia y seguridad en el almacenamiento y recuperación de datos\cite{MYSQL-oracle}. Además de poseer librerías ampliamente usadas para Java las cuales ayudan al desarrollo debilmente acoplado del \textit{back-end}.
        
        \subsection{SPRING}
        
        \subsection{Maven}
        
        \subsection{Hibernate}
        
        \subsection{iText}
        
    \section{Herramientas para el control de versiones y planificación}
    
    En la presente sección se presentan las herramientas de planificación y control de versiones usadas durante el desarrollo de la aplicación. Si bien no están vinculadas directamente al código, las mismas sirvieron de soporte para la organización del proyecto.
    
        \subsection{Git}
        
        Sistema de manejo de control de versiones, \textit{open source}, que provee la capacidad de crear tanto repositorios locales como remotos, ambos pueden estar sincronizados.
        
        \subsection{GitHub}
        
        Servidores online de repositorios remotos para Git. Puede ser usado de manera gratuita y pública. Permite el acceso a los repositorios de manera ininterrumpida y global.
        
        \subsection{Trello}
        
        Herramienta diseñada para la gestión de tareas usando listas o tablas. Cuenta con una interfaz intuitiva y de fácil aprendizaje.
        
        
        
    
\pagebreak