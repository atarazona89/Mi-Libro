\chapter*{Introducción}
\addcontentsline{toc}{chapter}{Introducción}

Actualmente en Venezuela los historiales médicos de los pacientes son llevados en papel por cada institución o sucursal de manera individual. Esto puede representar inconvenientes para los pacientes al momento de solicitar o requerir un cambio en la institución tratante, sea cual sea la razón. En estos casos los pacientes deben llenar los requisitos para la realización de un nuevo historial médico en la nueva institución y esto puede resultar en pérdidas de información por parte del nuevo médico tratante.

Globinsoft S.A. se dispuso a crear una plataforma web de alcance nacional y proyección internacional para la gestión integral de los historiales médicos que a su vez sirva de vínculo con las empresas farmacéuticas y farmacias en general. Con esta integración se busca facilitar la adquisición de medicamentos de parte de los pacientes y poder llevar un registro integral de los tratamientos, prescripciones y exámenes médicos a los que se haya sometido el paciente.

Según Lopcymat, se considera una enfermedad ocupacional los estados patológicos contraídos o agravados con ocasión del trabajo o exposición al medio en que el trabajador o trabajadora se encuentra obligado a trabajar. En este orden de ideas, se planteó la creación de ``HxPlus Ocupacional" como plataforma web de gestión de historias médicas orientado exclusivamente al área de medicina ocupacional.

Objetivo general: Desarrollar la primera versión de una aplicación que permita la gestión de historias médicas de los empleados de una empresa y automatice la generación de informes médicos a los doctores y departamento de recursos humanos.


Finalmente, el presente informe se compone de 6 capítulos que contienen: primero la descripción de la organización de Globinsoft S.A. así como información sobre proyectos previos realizados que sirven de precedente para ``HxPlus Ocupacional"; segundo la definición de los conceptos utilizados y manejados durante el desarrollo del proyecto; tercero la definición y estructura de la metodología utilizada para el desarrollo del proyecto, luego una descripción de las diferentes herramientas utilizadas, ya sea para el desarrollo como para la gestión del proyecto continuando con información detallada y puntual de las etapas de desarrollo de la aplicación y para terminar con las conclusiones generales.
%\end{enumerate}

%También cuenta con información adicional en el capítulo de apéndices, que puede servir como una visión alternativa de la aplicación o como un soporte adicional a lo descrito en los capítulos precedentes.

\pagebreak