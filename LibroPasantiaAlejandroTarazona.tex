\documentclass[letterpaper]{report}

\usepackage{polyglossia}
\setdefaultlanguage{spanish}



%\usepackage[spanish]{babel}
%\usepackage[utf8]{inputenc}

\usepackage{graphicx}
\usepackage{footnote}
\usepackage[top=2.5cm, left=3cm, right=3cm, bottom=2.5cm]{geometry}

\linespread{1.5}

\setlength{\parskip}{2ex}
\parindent 3ex

\usepackage[backend=biber,sorting=none]{biblatex}
\bibliography{Bibliografia/biblio}

\usepackage[linktocpage=true,hidelinks]{hyperref}

\DeclareGraphicsExtensions{.jpg,.pdf,.png}

\begin{document}	
     
     % Portada
     \begin{titlepage}
    \begin{center}
        
        % Upper part
        \includegraphics[scale=0.5]{figures/logo} \\
        
        \textsc {\large UNIVERSIDAD SIM\'ON BOL\'IVAR} \\
        \textsc{DECANATO DE ESTUDIOS PROFESIONALES\\
            COORDINACI\'ON DE INGENIER\'IA DE LA COMPUTACI\'ON}
        
        \bigskip
        \bigskip
        \bigskip
        \bigskip
        \bigskip
        \bigskip

        % Aqui coloca el nombre de su proyecto de grado/pasantia larga.
        \textsc{\bfseries HXPLUS OCUPACIONAL.\\
            SISTEMA DE GESTIÓN DE CONSULTAS MÉDICAS\\
             ORIENTADO A MEDICINA OCUPACIONAL}
        
        \bigskip
        \bigskip
        \bigskip
        \bigskip
        \vfill
        
        \begin{minipage}{\textwidth}
            \centering
            Presentado por: \\
            % Aqui coloca su nombre.
            Alejandro Tarazona \\
            
            Realizado con la asesoría de:\\
            
            Tutor Académico: Prof. Angela Di Serio\\
            
            Tutor Industrial: Ing. Juan Albarrán\\
            
            \bigskip
            \bigskip
            \vfill
            
            \textbf{INFORME DE PASANTÍA}\\
            
            Presentado ante la Ilustre Universidad Simón Bolívar\\
            
            como requisito parcial para optar por el título de\\
            
            Ingeniero en Computación
            
        \end{minipage}
        
        \bigskip
        \bigskip
        \vfill
        
         {\large \bfseries Sartenejas, \today}
    \end{center}
        
\end{titlepage}

\pagebreak
     % Título
     \input{titulo}
     
     % Copia y pega
     % Pagina de acta final (vacio)
     % Pagina del acta final
\begin{titlepage}
    \begin{center}
        
        % Upper part
        \includegraphics[scale=0.5]{figures/logo} \\
        
        \textsc {\large UNIVERSIDAD SIM\'ON BOL\'IVAR} \\
        \textsc{DECANATO DE ESTUDIOS PROFESIONALES\\
            COORDINACI\'ON DE INGENIER\'IA DE LA COMPUTACI\'ON}
        
        \bigskip
        \bigskip
        \bigskip
        \bigskip
        \bigskip
        \bigskip
        
        % Title
        \textsc{ACTA FINAL PASANTÍA LARGA}
        
        \bigskip
        \bigskip
        
        % Aqui coloca el nombre de su proyecto de grado/pasantia larga.
        \textsc{\bfseries HXPLUS OCUPACIONAL.\\
            SISTEMA DE GESTIÓN DE CONSULTAS MÉDICAS\\
            ORIENTADO A MEDICINA OCUPACIONAL}
        
        \bigskip
        \bigskip
        \bigskip
        \bigskip
        
        \begin{minipage}{\textwidth}
            \centering
            Presentado por: \\
            % Aqui coloca su nombre.
            \textsc{\bfseries Alejandro Tarazona} \\
            
            \bigskip
            \bigskip
            \bigskip
            \bigskip
            
            Este Proyecto de Pasantías ha sido aprobado por el siguiente jurado examinador: \\
            
            \bigskip
            \bigskip
            
            % Despues de cada line coloca el (los) nombre(s) de
            % cada uno de los integrantes del jurado.
            \line(1,0){200} \\
            Ángela Di Serio\\
            
            \bigskip
            \bigskip
            
            \line(1,0){200} \\
            Juan Albarrán\\
            
            \bigskip
            \bigskip
            
            \line(1,0){200} \\
            PROFESOR 3 \\
        \end{minipage}
        
        \bigskip
        \bigskip
        \vfill
        
        % Date/Fecha
        {\large \bfseries Sartenejas, \today}
        
    \end{center}
\end{titlepage}

\pagebreak
     
     \setcounter{secnumdepth}{3}
     \setcounter{tocdepth}{4}
     
     % Define encabezado numeros romanos y como se separan los captiulos y las
     % secciones
     \addtolength{\headheight}{3pt}
     \pagenumbering{roman}
%     \pagestyle{fancyplain}
     
     \renewcommand{\chaptermark}[1]{\markboth{\chaptername\ \thechapter:\,\ #1}{}}
     \renewcommand{\sectionmark}[1]{\markright{\thesection\,\ #1}}
     
%     \onehalfspacing
     
%     \lhead{}
%     \chead{}
%     \rhead{}
%     \renewcommand{\headrulewidth}{0.0pt}
%     \lfoot{}
%     \cfoot{\fancyplain{}{\thepage}}
%     \rfoot{}
     
     % Hasta aquí copié y pege
     
     \begin{center}
    \textbf{Resumen}
\end{center}

El presente libro describe el proceso de desarrollo empleado para la realización de una aplicación web orientada a medicina ocupacional, un área que es requerida desde hace pocos años por parte del Inpsasel; y que, además, representa una gran oportunidad para Globinsoft S.A., una empresa de desarrollo de \textit{software} que brinda servicios al área médica en el país.

Los \textit{framework} elegidos fueron AngularJS y Spring para \textit{front-end} y \textit{back-end} respectivamente, basándose los mismos en los lenguajes de programación Java y JavaScript usando el patrón de diseño Modelo Vista Controlador (MVC). Además se utilizó Git como sistema de manejo de versiones y GitHub como servidor del mismo.

Entre otras cosas, el sistema sigue la filosofía de Arquitectura Orientada a Servicios (SOA) y Autenticación Basada en Fichas (TBA).

El proyecto se llevó a cabo empleando la metodología SCRUM, basada en un proceso
iterativo de desarrollo incremental de las fases de especificación y análisis de requerimientos,
diseño, implementación y pruebas; y para llevar el control de dichas fases, se hizo uso de Trello, herramienta web que facilita el uso de listas de tareas.

Al culminar el proceso de desarrollo, se logró crear un sistema de gestión de consultas médicas que permite el almacenamiento de toda la información referente a las mismas, enfocado en medicina ocupacional, que permite realizar cruces de información tomando en cuenta los criterios de locación de trabajo y oficio al que se dedica el paciente y que permite realizar estudios de riesgo teniendo en cuenta los mismos. Todo esto para satisfacer el mercado naciente 

\pagebreak
     
     \tableofcontents
     
     % Índice de figuras
     % Crea la lista de cuadros
     %\listoftables
     
     % Crea la lista de figuras
     \listoffigures
     
     % Crea la lista de codigos fuentes
     %\lstlistoflistings
     
     \clearpage
     
     % Define encabezado en numeros arabicos  
     \pagenumbering{arabic}
     
     % Lista de símbolos y abreviaturas
     \chapter*{Introducción}
\addcontentsline{toc}{chapter}{Introducción}
\pagebreak
     \chapter{Entorno Empresarial}

En el presente capítulo se describe el entorno en el cual se desarrolló el proyecto de pasantía HxPlus Ocupacional, el cual fue realizado para la empresa Globinsoft S.A. Se presenta la historia, descripción, estructura organizacional y el cargo ocupado por el pasante dentro de la misma.

    \section{Antecedentes}
    Globinsoft S.A. posee en la actualidad su producto HxPlus el cual tiene como objetivo el almacenamiento de historias médicas de manera digital, usando tecnología web, para mantener su disponibilidad a cualquier hora del día y desde cualquier dispositivo con acceso a la web. Cuenta también con la opción de generar informes médicos, récipes y reposos médicos para uso de los farmacéutas y comodidad de los pacientes.
    
    
    \section{Misión}
    
    Brindar apoyo tecnológico al área médica en Venezuela, prestando servicios de calidad dentro del marco de lo estipulado por el Ministerio de Salud.
    
    \section{Visión}
    
    Ofrecer una plataforma integral para la gestión de consultas, historias médicas, récipes y medicamentos con alcance nacional y disponibilidad las 24 horas del día, los 7 días de la semana.
    
    \section{Estructura organizacional}
    
    Globinsoft S.A. mantiene los siguientes departamentos:
    
    \begin{enumerate}
        \item Gerencia.
        \item Recursos Humanos.
        \item Finanzas y Contabilidad.
        \item Proyectos.
    \end{enumerate}
    
    En la gerencia de proyectos se ubica el ingeniero Juán Albarrán, quien a su vez tiene el papel de tutor industrial de la pasantía descrita en el presente informe. La gerencia de proyectos se divide en cada uno de los proyectos realizados por la empresa y hasta el momento de la presente redacción, se cuenta con ``HxPlus" como proyecto en producción, al cual se le hace seguimiento, y soporte, y ``HxPlus Ocupacional" como proyecto en desarrollo. Figura \ref{estructura-org}.
    
    \begin{figure}[htbp!]
        \begin{center}
            \includegraphics[width=.8\textwidth]{figures/Estructura}
        \end{center}
        \caption{Estructura Organizacional de Globinsoft}
        \label{estructura-org}
    \end{figure}

\pagebreak
     \chapter{Marco Teórico}

En el presente capítulo se describen conceptos y patrones utilizados para el desarrollo del proyecto de pasantía, los cuales fueron seleccionados siguiendo los criterios de usabilidad, mantenibilidad, escalabilidad y portabilidad.

    \section{Modelo Vista Controlador (MVC)}
    
    El Modelo-Vista-Controlador es un patrón de arquitectura de \textit{software} que separa el modelo (objetos del negocio), la vista (interfaz con el usuario u otros sistemas) y el controlador (manejo de la información del negocio)\cite{MVC-tiw}.
    
    Técnicamente el usuario intereactúa con las vistas llenando formularios, solicitando información o simplemente haciendo algún \textit{click} que genere una petición al sistema. En ese momento es el controlador quien recibe la petición y genera las acciones necesarias sobre el modelo para así acceder a los datos y generar la nueva vista, resultado de la petición realizada y enviarla al usuario. Tal y como muestra la figura \ref{mvc-image}.
    
    \begin{figure}[htbp!] %Buscar imágen en español... -.-"
        \begin{center}
            \includegraphics[width=.7\textwidth]{figures/mvc}
        \end{center}
        \caption{Representación gráfica del patrón Modelo Vista Controlador\cite{MVC-imagen}}
        \label{mvc-image}
    \end{figure}
    
    Específicamente, cada componente tiene una asignación independiente de los demás componentes. Estas son:
    
    \begin{enumerate}
        \item \textbf{Modelo}
            \begin{itemize}
                \item Almacenamiento de los datos.
                \item Estado del sistema.
                \item Recuperación de errores a nivel de datos.
            \end{itemize}
        \item \textbf{Vista}
            \begin{itemize}
                \item Presentación del modelo.
                \item Accede al modelo pero no puede cambiar su estado.
            \end{itemize}
        \item \textbf{Controlador}
            \begin{itemize}
                \item Reacciona a las peticiones del cliente.
                \item Comunica al modelo de las acciones ejecutadas.
                \item Direcciona a las vistas requeridas del lado del cliente.
            \end{itemize}
    \end{enumerate}
    
    Esto se busca, primordialmente, para hacer del código altamente mantenible en el tiempo, ya que antiguamente se realizaban los sistemas siguiendo lo que se conoce como \textit{``programación de espaguetti"} (programación no estructurada) la cual no llevaba una separación entre lo que era la vista y los procesos internos del sistema. Esto traía como consecuencia, en el momento de realizar algún cambio al sistema ya sea de formato o de interacción, que tuviera que modificarse integralmente vista, interacciones y (potencialmente) el modelo de datos.
    
    \section{Arquitectura Orientada a Servicios}
    \label{teorico-soa}
    
    Es una filosofía de desarrollo la cual actúa como una guía de diseño y facilita el mismo orientado a la escalabilidad del sistema, facilita las actualizaciones de uno o varios de los componentes MVC y minimiza el efecto que dicha actualización tiene sobre los demás.
    
    Según \citeauthor{SOA-libroGartner}\cite{SOA-libroGartner} y \citeauthor{SOA-tesis}\cite{SOA-tesis}, la incorporación de SOA empieza en las empresas hacia 2003, por las siguiente razones:
    
    \begin{itemize}
        \item La incesante presión de los negocios para la agilidad. Cuando una empresa quiere
        modificar sus procesos, productos o servicios, no puede permitirse el lujo de esperar por
        mucho tiempo. Debe ser posible cambiar la forma de aplicación de los sistemas de
        trabajo simplemente modificando los componentes que ya están en uso, en lugar de
        comprar o codificar nuevos componentes o sistemas enteros desde cero.
        
        \item La flexibilidad de la arquitectura SOA basada en Servicios Web de apoyo a múltiples
        aplicaciones.
        
        \item  La aceptación unánime de proveedores de los estándares de Servicios Web,
        especialmente de Simple Object Access Protocol (SOAP) y Web Service Description
        Language (WSDL)\cite{SOA-libroGartner}
        
    \end{itemize}
    
    SOA se basa en capas, cada una ofrece servicios a la siguiente y sus procedimientos internos se mantienen ocultos a las demás capas. Con esto se generan APIs de acceso estandarizado que son independientes de las tecnologías utilizadas en el desarrollo.
    
    En \citetitle{SOA-msdn}\cite{SOA-msdn} definen SOA usando el poema de Saxe sobre seis ciegos y un elefante.
    
    ``Seis ciegos de Indostan se encuentran con un elefante, cada uno describe el elefante de forma diferente porque se ve influenciado por sus propias experiencias (Ver figura \ref{elefante-saxe})
    
    \begin{itemize}
        \item Quien le toca la trompa cree que es una serpiente.
        \item Quien le toca los colmillos cree que son lanzas.
        \item Quien le toca las orejas cree que son abanicos.
        \item Quien le toca la panza cree que es una pared.
        \item Quien le toca la cola cree que es una cuerda.
        \item Quien le toca	las patas cree que son árboles."
    \end{itemize}
    

    \begin{figure}[htbp!]
        \begin{center}
            \includegraphics[width=.8\textwidth]{figures/Elefante}
        \end{center}
        \caption {El elefante de Saxe}
        \label{elefante-saxe}
    \end{figure}

    Se usa la analogía para ejemplificar el hecho de que haya varias definiciones diferentes de lo que es SOA en si, porque se le ha definido como patrón de diseño o como una filosofía de desarrollo, siendo esta última la definición adoptada para el trabajo descrito en el presente informe.
    
    También se puede usar dicha analogía para ejemplificar cómo los (potenciamente diversos) dispositivos pueden interactuar con el controlador y este a su vez con el modelo sin que ello represente un cambio fundamental en diseño o estructura de los mismos. Cada dispositivo interactúa con los servicios que necesita y más ningún otro.

%    \section{Autenticación}
    

    \section{Autenticación Basada en \textit{Tokens}}
    
    Autenticación, según se lee en \cite{AUTENTICACION-rae}, es la ``acción y efecto de autentificar". Esto es comprobar ante una autoridad la veracidad o legitimidad de un documento o un hecho\cite{AUTENTICACION-wordref}.
    
    En sistemas, el término es utilizado para definir el proceso de verificación de las credenciales de usuarios dentro de un sistema. Comunmente usando el par ``nombre de usuario" y ``contraseña" para ello.
    
    Existen diversos métodos para llevar a cabo la autenticación\cite{AUTENTICACION-aplicaciones}:
    
    \begin{enumerate}
        \item Sistemas basados en algo conocido, ya sea \textit{password} o \textit{passphrase}.
        \item Sistemas basados en algo poseído que puede ser un tarjeta de identidad o dispositivos USB.
        \item Sistemas basados en características físicas como voz, huellas dactilares o patrones oculares.
    \end{enumerate}
    
    La autenticación basada en \textit{tokens} es una forma de autenticación ligera que va de la mano con SOA, usa \textit{tokens} (o fichas) cifrados para la verificación de usuarios. Estas fichas son almacenadas en el cliente y enviadas en cada una de los \textit{requests} que realiza el navegador, la ficha es descifrada y se verifican las credenciales del usuario en cuestión\cite{TOKEN-tokenbasedauth}.
    
    Entre los datos almacenados en las fichas suelen estar:
    
    \begin{itemize}
        \item Nombre de usuario.
        \item Fecha de autenticación.
        \item Fecha de caducidad de la ficha.
    \end{itemize}

    Estas son cifradas en el servidor bajo una clave secreta elegida por el mismo servidor y que se usa para el posterior descifrado de las fichas.
    
    Aunque no es un determinante, es deseable que la ficha contenga suficiente información del usuario para poder indentificarlo en cada una de las solicitudes y además sea lo suficientemente ligera como para no afectar la carga de datos. Todo esto permite que el servidor no se sobrecargue con variables de estado o de sesión por cada usuario autenticado en un momento dado y además permite la portabilidad desesada para el sistema.
    
    En la figura \ref{tba-request} se especifican los pasos básicos de la autenticación basada en \textit{tokens}:% y \ref{tba-server}.
    
    \begin{enumerate}
        \item El cliente hace una solicitud de autenticación enviando en la solicitud el nombre de usuario (\textit{username}) y su contraseña (\textit{password}).
        \item El servidor, una vez procesadas y confirmadas las credenciales del usuario, envía el \textit{token} cifrado de regreso para que sea almacenado en el cliente.
        \item El cliente envía una solicitud de lo que sería una página de ``inicio" dentro del sistema, adjuntando entre los encabezados (\textit{headers}) de dicha solicitud, el \textit{token} almacenado.
        \item El servidor, en caso de confirmar el \textit{token}, responde con la página y acciones solicitadas.
    \end{enumerate}
    
    \begin{figure}[htbp!]
        \begin{center}
            \includegraphics[width=.7\textwidth]{figures/tokenbarequest}
        \end{center}
        \caption{Autenticacion basada en fichas.}
        \label{tba-request}
    \end{figure}
    
    En adelante, y hasta que se complete el cierre de sesión, todas las solicitudes realizadas al servidor deben llevar adjunto el \textit{token} asignado para ser revisado en el servidor.
    
    En caso de fallos de autenticación, ya sea por credenciales incorrectas o un \textit{token} inválido, el servidor responderá con un mensaje de error y redireccionará la navegación a la vista de ``inicio de sesión" u otra página por defecto.
    
    \section{Inversión de Control}
    
    Concepto utilizado por primera vez por Martin Fowler\cite{IOC-dgarcia} que define una diferenciación principal entre un \textit{framework} y una biblioteca, siendo la biblioteca un conjunto de clases y funciones que pueden ser llamadas por el programa desarrollado y el \textit{framework} un diseño que controla el flujo y eventualmente llamará al código desarrollado para cumplir con las funciones específicas. De esta forma, el control de flujo no lo tiene el código si no el \textit{framework}, que responde a las solicitudes de los usuarios y solicita al programa las acciones requeridas.
    
    En este sentido, Spring cumple la cunción de manejar las solicitudes que surgen del \textit{front end} e invoca las secciones de código designadas como ``controladores" las cuales dependen (ver \ref{teorico-di}) de ``servicios".
    
    \section{Inyección de Dependencias}
    \label{teorico-di}
    
    Es una forma de IOC, se basa en que una clase A ``inyecte" objetos en otra clase B, siendo B dependiente de A, en lugar de permitir que la clase B se encargue de crear los objetos en sí misma\cites{IOC-mpachano}{IOC-dgarcia}.
    
    De esta forma se pueden crear objetos de una clase B y pasar el objeto de la clase A como argumento en el constructor de B o en algún \textit{setter} asignado al atributo. En adelante, cualquier función o procedimiento que precise información de A puede ser invocado a través de B aunque la responsabilidad de ejecución recaen en el objeto A.
    
    De forma más abstracta, se puede utilizar lo que en Java se conoce como \textit{interface} para dejar mayor margen a las modificaciones que puedan surgir con el tiempo en el código o en los requerimientos.
    
    En el caso de Spring, se usa inyección de dependencias en varios niveles:
    
    \begin{enumerate}
        \item Los ``controladores" utilizan a los ``servicios", que son interfaces creadas para realizar el procesamiento interno de los datos que llegan al servidor. Estas interfaces tienen procedimientos prederterminados, como toda ``interface", sus implementaciones deben implementar individualmente todos estos procedimientos y son inyectados (los ``servicios") en sus respectivos ``controladores" (figura \ref{img-controllerdi}).
        
        \begin{figure}[htbp!]
            \begin{center}
                \includegraphics[width=.8\textwidth]{figures/controller}
            \end{center}
            \caption{Ejemplo de inyección de dependencias en controladores.}
            \label{img-controllerdi}
        \end{figure}
        
        \item Los ``servicios" a su vez hace uso de los diferentes ``repositorios". También son interfaces que están estipuladas por JPA e implementadas por Hibernate (ver puntos \ref{tecno-jpa} y \ref{tecno-hibernate}) que son inyectados en los ``servicios" que así lo requieran (figura \ref{img-servicerdi}).

        \begin{figure}[htbp!]
            \begin{center}
                \includegraphics[width=.8\textwidth]{figures/serviceimpl}
            \end{center}
            \caption{Ejemplo de inyección de dependencias en servicios.}
            \label{img-servicerdi}
        \end{figure}        
        
    \end{enumerate}
    
\pagebreak
     \chapter{Marco Tecnológico}

    \section{Herramientas para el desarrollo de la aplicación}
    
    En esta sección se describen las herramientas usadas en el desarrollo de la aplicación y las características que hicieron que fueran seleccionadas para tal fin.
    
        \subsection{Eclipse}
    
        \subsection{Java}
        
        Lenguaje de programación utilizado para el desarrollo del \textit{backend}. Java es un lenguaje de programación imperativo orientado a objetos que facilita el desarrollo intependiente de los servicios y el fácil acceso al modelo de datos permitiendo así la baja cohesión entre los componentes del sistema.
        
        \subsection{JSON}
        
        Por \textit{JavaScript Object Notation}, es un formato de intercambio de datos, ligero\cite{JSON-yahoo} que permite una sencilla comunicación entre la vista y el controlador.
        
        "JSON es un formato de texto que es completamente independiente del lenguaje pero utiliza convenciones que son ampliamente conocidos por los programadores de la familia de lenguajes C, incluyendo C, C++, C\#, Java, JavaScript, Perl, Python, y muchos otros. Estas propiedades hacen que JSON sea un lenguaje ideal para el intercambio de datos\cite{JSON-jsonOrg}."
        
        Fue elegido por su compatibilidad con Java y porque es independiente de la tecnología usada en la Vista, lo cual permite a su vez realizar cambios en la Vista sin afectar las funcionalidades del controlador.
        
        \subsection{Angular JS}
        
        Es un \textit{framework} o conjunto de librerias orientada a facilitar el desarrollo web de aplicaciones dinámicas. "AngularJS le permite extender el vocabulario HTML para su aplicación"\cite{ANGULARJS-angularjs}.
        
        \subsection{JavaScript}
        
        JavaScript (abreviado comúnmente "JS") es un lenguaje de programación interpretado, dialecto del estándar ECMAScript. Se define como orientado a objetos, basado en prototipos, imperativo, débilmente tipado y dinámico\cite{JAVASCRIPT-wiki}.
        
        Es un lenguaje de programación orientado a crear contenidos dinámicos para páginas web del lado del cliente. Tanto JSON como AngularJS, previamente mencionados, usan JavaScript en su codificación.
        
        \subsection{MySQL}
        
        Manejador de bases de datos, \textit{open source}, ofrece alto rendimiento, eficiencia y seguridad en el almacenamiento y recuperación de datos\cite{MYSQL-oracle}. Además de poseer librerías ampliamente usadas para Java las cuales ayudan al desarrollo debilmente acoplado del \textit{back-end}.
        
        \subsection{SPRING}
        
        \subsection{Maven}
        
        \subsection{Hibernate}
        
        \subsection{iText}
        
    \section{Herramientas para el control de versiones y planificación}
    
    En la presente sección se presentan las herramientas de planificación y control de versiones usadas durante el desarrollo de la aplicación. Si bien no están vinculadas directamente al código, las mismas sirvieron de soporte para la organización del proyecto.
    
        \subsection{Git}
        
        Sistema de manejo de control de versiones, \textit{open source}, que provee la capacidad de crear tanto repositorios locales como remotos, ambos pueden estar sincronizados.
        
        \subsection{GitHub}
        
        Servidores online de repositorios remotos para Git. Puede ser usado de manera gratuita y pública. Permite el acceso a los repositorios de manera ininterrumpida y global.
        
        \subsection{Trello}
        
        Herramienta diseñada para la gestión de tareas usando listas o tablas. Cuenta con una interfaz intuitiva y de fácil aprendizaje.
        
        
        
    
\pagebreak
     \chapter{Marco Metodológico}

    En el presente capítulo se describe la metodología usada para el desarrollo del proyecto así como sus componentes, los actores que participaron y los roles que cumplieron en el desarrollo del mismo.

    Para \textit{HxPlus Ocupacional} fue seleccionado Scrum como metodología de desarrollo a seguir. Y aunque no está estipulado por la metodología, se llevaron a cabo diagramas de casos de uso y de clases para una documentación completa del sistema.
    
    Scrum es una metodología de gestión de proyectos ágil que utiliza uno o más equipos de trabajo, de a lo sumo 7 personas, en iteraciones de tiempo fijo, llamados \textit{Sprints}, para la entrega de tareas o avances en el proyecto que sean funcionales y probados
    \footnote{Con información de \citeauthor{scrum-guia}\cite{scrum-guia}, \citeauthor{scrum-primer}\cite{scrum-primer} y \citeauthor{scrum-agile}\cite{scrum-agile}}.    
    
    Los equipos apuntan siempre a conseguir avances limpios, probados y aceptados de manera que puedan ser puestos en producción inmediatamente.
    
    Scrum se divide en Roles, Eventos y Artefactos tal como se describe a continuación:

    \section{Roles}
        
        \subsection{Product Owner}
        \label{product-owner}
        
        Este rol representa la voz del cliente en la gestión del proyecto. Debe velar por la realización del proyecto desde la perspectiva del negocio. Se encarga de escribir las historias de usuario, las prioriza y las anexa al ``Product Balcklog" (o también Lista del Producto).
        
        El Dueño de Producto es la única persona responsable de gestionar la Lista del Producto. La gestión de dicha lista, según expresa \citeauthor{scrum-guia}\cite{scrum-guia}, consiste en:
        
        \begin{itemize}
            \item Expresar claramente los elementos de la Lista del Producto.
            \item Ordenar los elementos en la Lista del Producto para alcanzar los objetivos y misiones de la mejor manera posible.
            \item Optimizar el valor del trabajo desempeñado por el Equipo de Desarrollo.
            \item Asegurar que la Lista del Producto es visible, transparente y clara para todos, y que muestra aquello en lo que el equipo trabajará a continuación.
            \item Asegurar que el Equipo de Desarrollo entiende los elementos de la Lista del Producto al nivel necesario.
        \end{itemize}
               
        Para el presente proyecto se designó al ingeniero Juan Albarrán como Product Owner siendo el mejor representante de los intereses de Soluciones Globinsoft S.A. y estando él familiarizado tanto con los porcesos de negocio como con el equipo de desarrollo y el proceso de desarrollo mismo.
        
        \subsection{Scrum Master}
        \label{scrum-master}
        
        Es el responsable de llevar el procedimiento de gestión del proyecto según los lineamientos de Scrum. Sirve de asesoría entre invloucrados y comprometidos en materia de organización y distribución de la información. Él dirige las reuniones y vela por su cumplimiento según las reglas de procedimiento de Scrum.
        
        \citeauthor{scrum-guia}\cite{scrum-guia} clasifica los servicios del Scrum Master en tres categorías:
        \begin{enumerate}
            \item Servicio al dueño del producto
            \begin{itemize}
                \item Asesoramiento para la gestión eficiente de la Lista del Producto.
                \item Asesoramiento para la planificación del producto.
            \end{itemize}
            
            \item Servicio al equipo de desarrollo
            \begin{itemize}
                \item Apoyar en la organización del equipo.
                \item Eliminar los impedimientos y dificultades que limiten el progreso del equipo.
                \item Facilitar la organización de los eventos de Scrum según sea requerido.
                \item Guiar al equipo en entornos donde Scrum no haya sido del todo adoptado.
                \item Apoyar al equipo en dudas que surjan respecto a los eventos y artefactos de Scrum.
            \end{itemize}
            \item Servicio a la organización
            \begin{itemize}
                \item Liderar la adopción de Scrum como metodología de desarrollo.
                \item Asesorar a los empleados en el uso de Scrum y sus procedimientos.
            \end{itemize}
        \end{enumerate}
        
       Por su experiencia en previos proyectos, entre ellos \textit{HxPlus} (antecedente de \textit{HxPlus Ocupacional}) y su amplio conocimiento en el negocio, se le delegó también el puesto de ``Scrum Master" en el presente proyecto al ingeniero Juan Albarrán.
        
        \subsection{Development Team}
        \label{development-team}
        
        Es el conjunto de personas dedicadas a construir el producto indicado por el dueño del producto. Se requieren que sean grupos lo suficientemente pequeños como para fomentar un alto nivel de independencia y autoorganización pero, a su vez, lo suficientemente grandes como para poder presentar avances significativos en cada Sprint y que potencialmente, estos avances, puedan ser puestos en producción.
        
        Equipos pequeños, de menos de tres personas, reduce la productividad global y reduce los posibles avances en el producto. También podrían presentar limitaciones en cuando a las habilidades de los integrantes, lo cual resultaría convirtiéndose en impedimentos en el desarrollo.
        
        Por otro lado, equipos muy grandes, de más de nueve miembros requiere demasiada coordinación entre los miembros lo cual podría significar una menor eficiciencia y mayor trabajo para la organización de los avances. En otras palabras, se perdería agilidad tratando de organizar un equipo semejante.
        
        Idealmente un equipo de desarrollo varía entre tres y siete personas. Esto mantiene un equilibrio saludable entre la cantidad de trabajo que pueden manejar y la capacidad de autoorganización del equipo.
        
        El equipo debe ser multifuncional, debe tener las capacidades necesarias para desarrollar el producto y poder apoyarse entre sí para compensar los puntos débiles de cada individuo.
        
        No existen roles dentro del equipo de desarrollo. A pesar que uno podría desempeñarse mejor en un área de desarrollo, no existen como tal roles. Esto es debido a que la responsabilidad del desarrollo recae sobre el equipo como un todo y sobre ningún miembro en particular, por ello tampoco existe un rol de Líder o Jefe de Proyecto.
        
        Para HxPlus Ocupacional el equipo de desarrollo consta de un solo miembro, Alejandro Tarazona, pasante y autor del presente libro.
        
    \section{Eventos}
    
    Esta sección describe los eventos determinados por la metodología seleccionada y cómo fueron establecidos para la gestión del proyecto. Estos son:
    
    \begin{figure}[htbp!]
        \begin{center}
            \includegraphics[width=.8\textwidth]{figures/scrum}
        \end{center}
        \caption{Esquema de trabajo de SCRUM}
        \label{scrum-esquema}
    \end{figure}
    
        \subsection{Sprint}
        
        Unidad mínima de desarrollo, usualmente determinada por una tarea corta o un período de tiempo pequeño, usualmente 1 o 2 semanas, nunca más de 30 días; durante el cual el equipo de desarrollo trabaja según las metas estipuladas al principio del mismo. Normalmente éstas metas no cambian durante el desarrollo del Sprint sino al final del mismo, cuando se planifica el siguiente Sprint.
        
        Para HxPlus Ocupacional fue determinado en una semana para las primeras tareas y dos para las últimas, debido a las pruebas subyacentes y el trabajo de integración que representan.
        
        Se llevó a cabo 3 fases:
        \begin{enumerate}
            \item Preparación (1 Sprint)
            \item Implementación (8 Sprints)
            \item Cierre (2 Sprint)
        \end{enumerate}
        
        Tal y como se describe en el capítulo \ref{desarrollo-capitulo}.
        
        \subsection{Sprint Planning}
        
        Planificación del siguiente Sprint a realizar. El equipo de desarrollo (``Development Team", punto \ref{development-team}) hace los pronósticos e indica qué puede llevar a cabo en el siguiente Sprint de lo que propone el dueño del producto (``Product Owner", punto \ref{product-owner}) que debe hacerse durante el Sprint. El ``Scrum Master" (punto \ref{scrum-master}) está encargado de revisar cuidadosamente con el equipo las posibilidades, negociar con el dueño del producto las posibilidades de realización y establecer las metas.
        
        Semanalmente se realizó una reunión del \textit{Development Team} con el \textit{Scrum Master} y \textit{Product Owner} para evaluar el resultado del Sprint de esa semana y realizar la planificación adecuada a los logros y el desenvolvimiento en el proyecto.
        
        \subsection{Daily Sprint Meeting}
        
        Reuniones diarias realizadas entre el ``Scrum Master" y el equipo de desarrollo para revisar los avances diarios, aclarar dudas y difundir información acerca del progreso alcanzado hasta el momento. Tiempo fijo en 15 minutos y usualmente se realizan en la mañana.
        
        En este punto, Scrum, hace una diferenciación clave entre los elementos que están compromentidos (\textit{``commited"}) y los que sólo están involucrados (\textit{``involved"}) en el desarrollo del proyecto. Siendo comprometidos los equipos de desarrollo, el ``Scrum Master" y el ``Product Owner" e involucrados los demás departamentos de la empresa que puedan tener interés en el estado del proyecto (dpto. de ventas, clientes, etc). 
        
        En este órden de ideas, durante una reunión diaria de Sprint, sólo los comprometidos tienen potestad de hablar o comentar las cosas que han sucedido. Esto se hace para lograr que, en los 15 minutos de duración de la reunión, se discutan temas que sean de suma necesidad para el desarrollo del proyecto, se ponen de manifiesto dificultades técnicas o impedimentos dentro de los equipos de desarollo y, también, para difundir información sobre el estado del proyecto a las partes involucradas.
        
        La responsabilidad de resolver todo impedimento manifestado en dichas reuniones recae sobre el ``Scrum Master".
        
        Durante el proyecto se realizaron las reuniones con la presencia del Ing. Juan Jesús Albarrán para actualizar el estado del desarrollo del proyecto y aclaración de dudas por parte del pasante en cuanto a lo acaecido durante el día previo.
        
        \subsection{Sprint Review}
        
        Son reuniones que se realizan, como su nombre lo indica, para hacer una revisión del trabajo realizado durante el Sprint y presentar el trabajo completado a las partes involucradas. Estas reuniones no deben pasar de 4 horas de duración y todo trabajo incompleto no debe ser presentado.
        
        Una vez mostrados los resultados del Sprint, el ``Product Owner" debe realizar una evaluación de las metas cumplidas (o no) y si ha habido cambios en el contexto, deberá también realizar las adaptaciones necesarias a la planificación del proyecto.
        
        En \textit{HxPlus Ocupacional} se realizaoron en conjunto las reuniones de \textit{Sprint Planning} y \textit{Sprint Review} del último Sprint finalizado con la finalidad de minimizar el tiempo de reuniones y aprovechar las disponibilidades de los comprometidos y los involucrados.
        
        \subsection{Sprint Retrospective}
        
        Al finalizar cada Sprint el equipo se reúne con un tiempo fijo de 4 horas para revisar sus técnicas y la forma en que han abordado el desarrollo del proyecto, discutir las impresiones referentes al Sprint superado y revisar los inconvenientes presentados.
        
        Es deber del ``Scrum Manager" revisar los inconvenientes y buscarles solución rápida para mejorar la productividad del equipo.
        
        Debido al que el grupo de trabajo sólo consta de 1 desarrollador, se consideró inconveniente realizar reuniones de 4 horas exclusivamente para hacer la restrospectiva. En su lugar se atendieron los inconvenientes, dudas y revisones durante las reuniones diarias y se apartó un espacio de 15 minutos de las reuniones de planificación y revisión para realizar actividades de ésta reunión.
        
    \section{Artefactos}
    
    Documentos realizados para llevar registro de las etapas de desarrollo del proyecto y a su vez realizar la evaluación de las mismas.
    
        \subsection{Product Backlog}
        
        O también ``Lista de Producto". Es una lista con todas las consideraciones necesarias de parte del dueño del producto, quien la organiza y la gestiona. Cualquier cambio a realizarse dentro de la planificación debe pasar por esta lista.
        
        En esta lista se enumeran los deseos del cliente, se priorizan y se estima el esfuerzo requerido. Esta lista debe ser seguida por el equipo de desarrollo para dirigir sus avances.
        
        Es una lista que hace el Dueño del Pruducto iniciando el proceso de gestión de requerimientos, sin embargo, esta lista tiene la característica de ser mutable, como los requerimientos del Dueño del Producto, y es modificada conforme sea necesario o requerido. Por eso es que \citeauthor{scrum-guia}\cite{scrum-guia} dice: ``Una Lista de Producto nunca está completa".
        
        Para ello existe el ``refinamiento" de la lista del  producto, el cual  es el proceso de añadir detalles, granularidad y prioridad a cada uno de los requerimientos, según \citeauthor{scrum-guia}\cite{scrum-guia}. Usualmente, las tareas o requerimientos que pasan a ser parte de la siguiente planificación de Sprint son las de mayor prioridad y granularidad y que, además, suele ser el caso que las actividades prioritarias son refinadas primero para así llevarlas a desarrollo lo antes posible.
        
        En el caso de \textit{HxPlus Ocupacional} el Dueño del producto realizó un levantamiento de requerimientos y utilizó ``Trello" para la gestión de los requerimientos.
    
        \subsection{Sprint Backlog}
        
        O ``Lista de Pendientes del Sprint". Es una lista de objetivos del Sprint tomada de la lista de producto durante la planificación del Sprint. Puede ser uno o varios objetivos, lo suficientemente refinados como para que el equipo de desarrollo pueda entenderlos en la reunión diaria y puedan ser llevados a cabo durante el Sprint.
        
        Según se requiera nuevo trabajo, el equipo de desarrollo lo irá añadiendo a la lista de pendientes del Sprint, ya sea por inconvenientes surgidos o problemas no tomados en cuenta o por refinamiento de los objetivos. Además, conforme el trabajo vaya siendo completado, se debe actualizar la estimación del trabajo restante. Sólo el equipo de desarrollo tiene potestad sobre la lista de pendientes del Sprint y es su forma de ver, transparentemente y en tiempo real, el estado del dearrollo de un Sprint.
        
        Usando también las facilidades de ``Trello", el equipo de desarrollo gestionó cada Sprint a través de las listas creadas dentro de una ``Pizarra" del sistema.
    
\pagebreak
     \chapter{Desarrollo de la Aplicación}
    \section{Primer Sprint}
        \subsection{Objetivos}
        Levantar los requerimientos del producto.
        \subsection{Actividades}
        \begin{itemize}
            \item Creación del \textit{Product Backlog}
            \item 
        \end{itemize}
    \section{Segundo Sprint}
        \subsection{Objetivos}
        \begin{itemize}
            \item Documentar la aplicación a realizar.
            \item Introducción al ambiente de trabajo.
        \end{itemize}
        \subsection{Actividades}
        \begin{itemize}
            \item Creación del diagrama ER-E para la base de datos de la aplicación.
            \item Creación de un modelo de casos de uso de la aplicación.
            \item Descarga de las herramientas ya mencionadas en el marco tecnológico.
            \item Creación de repositorios, tanto local como remoto, para el almacenamiento de la aplicación.
        \end{itemize}
        
    \section{Tercer Sprint}
        \subsection{Objetivos}
        \begin{itemize}
            \item Configuración de \textit{SPRING} e \textit{Hibernate}
        \end{itemize}
        \subsection{Actividades}
        \begin{itemize}
            \item Configuración de las características de \textit{SPRING} para trabajar con anotaciones.
            \item Descargar y configurar las librerías internas de la herramienta.
            \item Descargar la librería JSON para la comunicación con el \textit{frontend}.
            \item Descargar y configurar \textit{Hibernate} para la comunicación con la base de datos.
        \end{itemize}
        
    \section{Cuarto Sprint}
        \subsection{Objetivos}
        \subsection{Actividades}
        
    \section{Quinto Sprint}
        \subsection{Objetivos}
        \subsection{Actividades}
        
    \section{Sexto Sprint}
        \subsection{Objetivos}
        \subsection{Actividades}
        
    \section{Septimo Sprint}
        \subsection{Objetivos}
        \subsection{Actividades}
        
    \section{Octavo Sprint}
        \subsection{Objetivos}
        \subsection{Actividades}
        
    \section{Noveno Sprint}
        \subsection{Objetivos}
        \subsection{Actividades}
        
    \section{Sprint}
        \subsection{Objetivos}
        \subsection{Actividades}
        
    \section{Dificultades generales encontradas}
    \section{Resultados}
\pagebreak
     \chapter*{Conclusiones y Recomendaciones}
\addcontentsline{toc}{chapter}{Conclusiones y Recomendaciones}    
   
     \addcontentsline{toc}{chapter}{Anexos}
\pagebreak
    % Diagrama CU
    % Diagrama Clases
    % Diagrama ER-E
    % Screenshots
    % DAS(?)
     \addcontentsline{toc}{chapter}{Bibliografía}
\printbibliography
\pagebreak
     
\end{document}