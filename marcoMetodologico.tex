\chapter{Marco Metodológico}

    En el presente capítulo se describe la metodología usada para el desarrollo del proyecto así como sus componentes, los actores que participaron y los roles que cumplieron en el desarrollo del mismo.

    Para \textit{HxPlus Ocupacional} fue seleccionado Scrum como metodología de desarrollo a seguir, la cual se divide en Roles, Eventos y Artefactos tal como se describe a continuación:

    \section{Roles}
        \subsection{Scrum Master}
        Juan Albarrán. Representante de Globinsoft S.A. y líder del proyecto \textit{HxPlus Ocupacional}.
        
        \subsection{Product Owner}
        Juan Albarrán.
        
        \subsection{Development Team}
        Alejandro Tarazona. Pasante encargado del desarrollo del proyecto y autor del presente libro.
        
    \section{Eventos}
    
    Esta sección describe los eventos determinados por la metodología seleccionada y cómo fueron establecidos para 
    
        \subsection{Sprint}
        
        Unidad mínima de desarrollo, usualmente determinada por una tarea corta o un período de tiempo pequeño. Para HxPlus Ocupacional fue determinado en una semana para las primeras tareas y dos para las últimas, debido a las pruebas subyacentes y el trabajo de integración que representan.
        
        \subsection{Sprint Planning}
        
        Semanalmente se realizó una reunión del \textit{Development Team} con el \textit{Scrum Master} y \textit{Product Owner} para evaluar el resultado del Sprint de esa semana y realizar la planificación adecuada a los logros y el desenvolvimiento en el proyecto.
        
        \subsection{Daily Sprint Meeting}
        
        Diariamente se realizaron reuniones para evaluar el avance durante el día en cuestión y la aclaratoria de dudas puntuales que fueron surgiendo en el desarrollo.
        
        \subsection{Sprint Review}
        
        Junto con las reuniones de \textit{Sprint Planning} se realizó la evelauación o \textit{Review} del Sprint de la semana en cuestión.
        
        \subsection{Sprint Retrospective}
        
    \section{Artefactos}
    
    Documentos realizados para llevar registro de las etapas de desarrollo del proyecto y a su vez realizar la evaluación de las mismas.
    
        \subsection{Sprint Backlog}
        
        Por cada Sprint se realizó una reunión de planificación y una de evaluación, las conclusiones, para cada sprint, se ven reflejadas en el \textit{Sprint Blacklog}.
            
        \subsection{Product Backlog}
        
        La colección de todos los \textit{Sprint Backlog}, genera el \textit{Product Backlog}. En él se puede evaluar el desarrollo del proyecto de manera global.
        
    
\pagebreak