\begin{center}
    \textbf{Resumen}
\end{center}

El presente libro describe el proceso de desarrollo empleado para la realización de una aplicación web orientada a medicina ocupacional, un área que es requerida desde hace pocos años por parte del Inpsasel; y que, además, representa una gran oportunidad para Globinsoft S.A., una empresa de desarrollo de \textit{software} que brinda servicios al área médica en el país.

Los \textit{framework} elegidos fueron AngularJS y Spring para \textit{front-end} y \textit{back-end} respectivamente, basándose los mismos en los lenguajes de programación Java y JavaScript usando el patrón de diseño Modelo Vista Controlador (MVC). Además se utilizó Git como sistema de manejo de versiones y GitHub como servidor del mismo.

Entre otras cosas, el sistema sigue la filosofía de Arquitectura Orientada a Servicios (SOA) y Autenticación Basada en Fichas (TBA).

El proyecto se llevó a cabo empleando la metodología SCRUM, basada en un proceso
iterativo de desarrollo incremental de las fases de especificación y análisis de requerimientos,
diseño, implementación y pruebas; y para llevar el control de dichas fases, se hizo uso de Trello, herramienta web que facilita el uso de listas de tareas.

Al culminar el proceso de desarrollo, se logró crear un sistema de gestión de consultas médicas que permite el almacenamiento de toda la información referente a las mismas, enfocado en medicina ocupacional, que permite realizar cruces de información tomando en cuenta los criterios de locación de trabajo y oficio al que se dedica el paciente y que permite realizar estudios de riesgo teniendo en cuenta los mismos. Todo esto para satisfacer el mercado naciente 

\pagebreak